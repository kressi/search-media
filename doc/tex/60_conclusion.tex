\chapter{Fazit}
In dieser Arbeit wurden zuerst die Formate der Mediadateien
und deren Container für Tags analysiert. Es ist interessant
die nicht geringen Unterschiede zwischen unterschiedlichen
Formaten zu erkennen. Genauso ist es spannend zu sehen,
dass ID3 Tags nicht nur für das MP3 und Vorbis Comments nicht
nur für FLAC Format verwendet werden. Sobald man jedoch damit
beginnt die Metadaten mit der Apache Tika Library zu parsen,
stellt man fest, dass zumindest im Bereich von weit verbreiteten
Mediaformaten, auch ohne jegliches Wissen über die verwendeten
Tag Container oder deren Aufbau weiterkommt. Selbst mit
Wissen über die Tag Container ist es einfacher den
\emph{AutoDetectParser} zu verwenden. Dies macht es
unglaublich leicht Metadaten aus Dateien zu extrahieren.

Als nächster Schritt wurden die Metadaten indiziert.
Auch dieser Schritt geht mit den Apache Libraries leicht von
der Hand, zumindest solange man die Standardkonfiguration verwendet.
Dies ist sehr angenehmn, so sieht man schnell
ein Resultat, was motiviert weiterzumachen. Genauso
sieht man schnell ein Resultat wenn man eine Abfrage erstellt.

Für die Analyse wurde das Tool Luke verwendet, der Funktionsumfang
dieses Tools ist für die Entwicklung und die Analyse von
Indizes unglaublich wertvoll. Um jedoch einmal erst
soweit zu kommen hat es sich erstaunlich kompliziert
herausgestellt eine Version von Luke zu finden, welche
die aktuellste Version\footnote{Zum Zeitpunkt dieser Arbeit war 6.0.0
die aktuellste stabile Version von Apache Lucene.} des
Lucene Index lesen kann. Dies mag daran liegen, dass
normalerweise \href{https://www.elastic.co/products/elasticsearch}{Elasticsearch}
oder \href{https://lucene.apache.org/solr/}{Solr} verwendet
werden, welche gleich die eigenen Analyse Tools mitliefern.
Bei der Analyse selbst wurden keine so hohe Precision und Recall
Werte erwartet. Wenn man jedoch das Information Retrieval System
betrachtet ist es logisch solche Werte zu erreichen.

