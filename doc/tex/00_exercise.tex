\chapter{Aufgabenstellung}
\paragraph{Ausgangslage} \hfill \\
Auf einem privaten Network Attached Storage (NAS) sind mehrere Hundert Video und mehrere Tausend Musikdateien abgelegt. Diese sind in groben Ordnerstrukturen organisiert. Clientseitig existiert eine Suche, mit welcher die Dateinamen durchsucht werden können. Da die meisten dieser Dateien mit Metadaten versehen ist es wünschenswert auch diese Daten durchsuchen zu können.

\paragraph{Ziel der Arbeit} \hfill \\
Im Rahmen dieser Arbeit sollen die Metadaten der Dateien einer Media Library auf einem Network Attached Storage indiziert werden. Auf dem erstellten Index sollen Suchabfragen ausgeführt werden können.

\paragraph{Aufgabenstellung} \hfill \\
\begin{itemize}
    \item Vorbereiten der Suche
        \begin{itemize}
            \item Es sollen mögliche Medienformate untersucht werden.
        \end{itemize}
    \item Aufsetzen der Suche
        \begin{itemize}
            \item Indizieren von ein bis zwei der untersuchten Medienformaten.
            \item Suchabfragen erstellen.
        \end{itemize}
    \item Analysieren der Suchabfragen
        \begin{itemize}
            \item Testdaten bestimmen.
            \item Qualität der Suche anhand von Precison und Recall mit den Testdaten bestimmen.
        \end{itemize}
\end{itemize}

\paragraph{Erwartete Resultate} \hfill \\
\begin{itemize}
    \item Analyse der möglichen Medienformate und Entscheid für ein bis zwei Formate für die Indizierung.
    \item Ein lauffähige Instanz von Lucene.
    \item Vorbereitung der Dokumente (Mediadateinen) um die Metadaten zu extrahieren.
    \item Definition von Beispiel-Anfragen und den dazu erwarteten Antworten.
    \item Analyse der Resultate.
\end{itemize}
